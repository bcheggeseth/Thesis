\documentclass[11pt]{article}
\usepackage{fullpage,amsmath,amsfonts,graphicx,amsthm}
\usepackage{setspace}
\title{Data Application}
\author{Brianna C. Heggeseth}

\newtheorem{theorem}{Theorem}
  \newcommand{\B}[0]{\mathbf}
    \newcommand{\bs}[0]{\boldsymbol}

\begin{document}
\doublespace
\maketitle
\section{Data}
The Center for the Health Assessment of Mothers and Children of Salinas (CHAMACOS) study is a longitudinal birth cohort designed to assess the health effects of pesticides and other environmental exposures on the growth and development in children living in the agricultural Salinas Valley, CA \cite{eskenazi2004,eskenazi2005}. Pregnant women were recruited in 1999-2000 in prenatal clinics, with 528 mother-child pairs in the study at delivery and 327 pairs remaining at the 9-year visit. Baseline maternal characteristics such as height and weight were measured at the start of the study and maternal urine and blood samples were taken twice during pregnancy and then again at delivery to measure levels of pesticide and chemical exposure. Child leptin levels were measured at interviews that occurred at birth and approximately 2 years, 5 years, and 9 years for a convenient sample of the children. For this paper, we limit our analysis to 80 children who have leptin levels available for all of the four time points. Details of the study are published elsewhere \cite{eskenazi2003}. All study activities were approved by the Committee for the Protection of Human Subjects at the University of California, Berkeley. \\\\
Leptin, a hormone synthesized primarily by adipose tissue but also by other major organs, acts on the hypothalamus to convey satiety and regulate long-term energy balance (cite Green 1995, Margetic 2022, Koerner 2005, Mantzoros 2001). High levels of leptin signal a sensation of being full and content, while low levels result in feelings of hunger and craving more food. Since the hormone is produced by adipose (fat) tissue, levels are proportion to amount of body fat; obese individuals have higher level of leptin, which is thought to cause leptin resistance and decreased sensitivity of the hypothalamus to changes in leptin levels. Very few studies have repeatedly measured leptin levels of individuals to study the change of leptin levels during early childhood, which is thought to be a critical obesity development period. In the cases in which more than one measurement is taken, leptin is observed at birth and then once more within a few years in an attempt to be able to predict leptin and adiposity levels at the later age. Characterizing leptin growth trajectories of children and whether there are distinct patterns is of great importance to public health officials since hormone levels are directly related to adiposity tissue and energy balance. We will use this data set to illustrate the differences between clustering methods and how they work in practice with real data sets.\\\\
We use the standard and novel methods described in this thesis and compare the clustering results and the inference on baseline variables that may be related to cluster membership. First, as mentioned in the implementation sections of Chapter 4, we need to make decisions about the B-spline basis used in many of the methods. Figure \ref{fig:leptin} shows the original data. There are 80 lines corresponding to the four observed leptin levels for the 80 children in the data set. NOTE THAT INDIVIDUALS WERE NOT INTERVIEWED EXACTLY ON THEIR BIRTHDAY SO THERE IS VARIABILITY IN THE AGE OF INTERVIEWS WITHIN THE SAMPLE. ONE COULD MAKE THE ARGUMENT TO IGNORE THIS VARIABILITY AND COLLAPSE THE AGES OF INTERVIEWS IN MONTHS TO THE DESIRED AGE OF INTERVIEW IN YEARS SO THAT WE HAVE A VECTOR OF LENGTH 4 FOR EACH INDIVIDUAL. THE CHOICE OF THE TIME VARIABLE AFFECTS THE DEGREES OF FREEDOM THAT CAN BE USED IN THE BASIS SINCE IT IS LIMITED BY THE RANK OF THE DESIGN MATRIX OF BASIS VARIABLES. IF WE USE THE FOUR TIME POINTS, WE CAN ONLY HAVE AT MOST 4 DEGREES OF FREEDOM FOR THE BASIS SO THAT $X^{T}X$ IS INVERTIBLE. ON THE OTHER HAND, IF WE USE THE AGE AT INTERVIEWS, WE ARE NOT AS LIMITED IN THIS WAY. Based on the visual inspection of the data, we determine that quadratic curves with one knot should be sufficient to model the complexity given our data limitations. ONE PLACE OF DRASTIC CHANGE AND WE DON'T HAVE ENOUGH DATA POINTS PER INDIVIDUAL TO REALLY ESTIMATE A HIGHER ORDERED POLYNOMIAL CURVE WITHOUT EXTRAPOLATING QUITE A BIT BETWEEN POINTS.
\begin{figure}
\begin{center}
Figure with original data
\end{center}
\label{fig:leptin}
\caption{Logarithm of leptin measurements for 80 children in the CHAMACOS data observed at birth and approximately 2 years, 5 years, and 9 years of age. }
\end{figure}
Where to place the one internal knot. First, we can place it at the median time points. That forces a curvature to the later parts of the trajectory and forces the low point of the mean curve to be at the median, where we don't have any data. I would be better to have more smooth linear interpolation between the points so as to not extrapolate too much beyond the scope of the data. For both data sets, there is a drastic change in the shape that occurs at around the observation time of 24 months. This pulls the low point to the observe data and allows the rest of the curve to better fit the data.

Using the corresponding techniques to select the number of clusters, we perform all types of cluster analysis on the data set. 
GRAPHS FOR EACH METHOD
TABLE WITH ESTIMATES FOR KEY CONCOMITANT VARIABLES FOR EACH METHOD

\end{document}