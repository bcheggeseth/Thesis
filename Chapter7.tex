\chapter{Conclusions}
\label{chap:concl}
Longitudinal data sets include measurements repeated at irregularly intervals over time on a relatively large number of subjects. This highly structured data requires methods that account for the time-ordering and dependence between measurements on the same subject. I focus on cluster analysis methods, propose new methods to group longitudinal data by the pattern of outcome change over time, and compare the performance of standard clustering methods with the proposed in answering research questions about shape. 

\section{Contributions}
This thesis makes several original contributions. The first major contribution is my study of covariance misspecification in mixture models in Chapter \ref{chap:misspecify}. When components of a mixture model overlap, it is important to correctly model the within-vector dependence to avoid asymptotic and finite sample bias in the parameter estimates when the number of components are known. Naively assuming conditional independence may bias the clustering results that can lead to incorrect conclusions. This is contrast to most longitudinal data analysis in which the mean can be estimated in an unbiased manner without modeling the inherent dependence with robust standard error estimate for inference. Using a working correlation close to the truth improves the statistical efficient of the estimator. Although not studied in this thesis, covariance misspecification also impacts the choice of the number of optimal components when it is not known a priori. If the dependence structure is more complex than the assumed model, more components are needed to model the variability and dependence. This has practical importance when fitting a mixture model to longitudinal data as there is inherent dependence within subjects. This adds to and is consistent with the limited literature on misspecifying mixture in that well separated components are more robust to misspecification. 

When clustering longitudinal data, it is important to consider how two trajectories are deemed similar. Longitudinal studies are typically carried out to directly study the change over time. The second contribution of this thesis is raising awareness that most clustering algorithms do not group individuals based on how trajectories change over time.  Rather, most methods including the standard finite mixture model groups individuals on the feature that dominates the variability, which is often the overall outcome level. The few methods that have been suggested as a way to cluster based on shape are limited as they only flourish in certain circumstances.

The third contribution is adapting and extending standard methods to attempt to cluster based on shape while overcome the shortfalls of the current methodology. The three proposed methods approach the problem from different angles. All of the methods have issues and challenges, but vertically shifted mixture models outperforms them when in clustering short noisy trajectories.

Lastly, this thesis studies growth trajectories by clustering individuals into distinct groups and estimating the relationship between baseline factors and group membership. The proposed vertically shifted mixture model is fit to the CHAMACOS BMI outcome data and the results are juxtaposed with those from standard clustering methods that are currently used in the literature. In this data set, level and shape are moderately related so the group mean curves are similar between clustering methods, but the groups differ enough in composition to impact the direction and inference of the relationship with baseline factors. Maternal pre-pregnancy BMI is significantly associated with growth trajectory groups primarily determined by level but not necessarily with those based on shape. On the other hand, this study provides some moderate evidence to suggest that prenatal exposure to BPA may impact the rate of growth but not necessarily the level of BMI.

\section{Limitations}
While this thesis attempts to be thorough in its study of these clustering methods, it is not exhaustive. The study on covariance misspecification primarily focuses on wrongly assuming conditional independence when the true data generating dependence for one of the groups is exchangeable with constant variance. There are infinitely many simulation possibilities for generating and estimating covariance models. Additionally, the trends used to generate the data were limited to horizontal lines over time knowing that more complex shapes translate into more well separated components in general. To fully understand the impact of misspecification, one could also use different mean shapes, more components, non-stationary generating covariance structures and varying vector lengths. However, the study highlighted key issues of assuming independence for longitudinal data, choosing a covariance model, and the impact of component separation on clustering. More suggestions for more simulation studies are listed at the end of Chapter \ref{chap:misspecify}. 

The context in which I discuss and compare different clustering methods is a longitudinal data with only five to ten repeated outcome measurements sparsely observed over time, which is common for the field of public health. Some of the methods may work better on longer trajectories. In that case, if the data is sampled on a dense grid, there may be more options available in the functional data analysis literature. While the restricted scope may be considered a shortfall of this thesis, I believe it is worth focusing our attention on this type of data set since it is so common in practice.

In terms of the proposed methods, the vertically shifted mixture models excels in practice even though there are theoretic issues. By subtracting the mean, the transformed vector lies on a subspace of one dimension smaller and thus the covariance matrix of the vertically shifted vector is always singular no matter the original dependence structure. However, we need to model the covariance of the deviations from the mean for the transformed vectors. If the observation times are the same for every individual, the matrix is singular and regularization of the covariance matrix is necessary for estimation. When this is the case, the vertically shifted model may not be able to accurately detect subtle shape differences due to the misspecification in the covariance matrix. If the data are observed at random times, the transformation may have a valid covariance matrix and modeling may be more accurate. 

Many believe that the gold standard for longitudinal studies is to measure individuals at evenly spaced, regular times. However, there is no scientific reasoning for this uniform temporal design \cite{collins2006}. Rather, the temporal design should be based on the expected trajectory shape; more observations are needed during times of rapid or non-linear change. Work in this thesis suggests a random temporal design is may be preferable when vertically shifting the data, but it is also important to have enough data points throughout the follow-up period to thoroughly observe individual trajectories. Therefore, a temporal design should be chosen based on the expected shape plus some systematic or random variability in the actual measurements times between individuals. 

\section{Future work}
In addition to answering questions, the work in this thesis creates many new questions. Mixture models are sensitive to assumptions and incorrectly modeling the covariance structure can lead to bias. However, it is hard to know whether there is bias when working with real data. For most statistical models, resampling methods can be used to estimate the magnitude of the bias; however, implementation for mixture models is not straightforward. \Textcite{grun2004} provided suggestions for bootstrapping finite mixture models as a diagnostic tool but did not discuss the issue of bias. This is an area needing more work as practitioners need some indication that the results are potentially misleading due to bias. 

While this thesis has focused on longitudinal data with few time observations, some of the clustering methods discussed in this thesis may perform better when there is more data points per person. Future work should include increasing the number of observations when comparing the three proposed methods. Additionally, as the number of observations increases, the line that distinguishes longitudinal data from functional data is blurred. It is worth determining the point at which functional clustering algorithm may be more applicable than those proposed in this thesis. It is also worth investigating the possibility of extending these ideas of shape to non-continuous outcome data \cite{jones2001} as well as multivariate longitudinal trajectories \cite{jones2007,d2000}.

I have limited the discussion to only include basis functions of time in the design matrix for the mean regression, but the model is general enough to allow other covariates. However, including time-fixed covariates that only impact the vertical level is futile when clustering based on shape. If these covariates are thought to impact the shape, one can include them in the generalized logit design vector so as to estimate the association with shape group membership. On the other hand, time-varying covariates can be used to model the mean trajectory shape. However, the interpretation of the estimated coefficients in the context of clustering is not straightforward. \Textcite{nagin2003} suggested using time-varying binary covariates to model abrupt or sharp changes in individual trajectories caused by life course turning points. However, if these binary covariates change states multiple times during the follow-up period, a hidden Markov model or switching regression models \cite{quandt1972} may be more appropriate. Future work needs to consider the interpretation of shape clusters when continuous time-varying covariates are introduced into the mean structure changes the  as groups are also formed on the basis of the relationships with these new covariates. 

One of the largest contributions of this thesis is separating shape and level into two distinct characteristics of longitudinal trajectories. Research questions usually focus on one or the other so it is important to separate these features so as to not muddle the interpretations when doing clustering analysis. One suggestion for systematically studying both features is to complete a pseudo two-part model in which two mechanisms impact the outcome values. The standard two-part model views a semicontinuous univariate outcome response as the result of two processes, one determining whether the outcome is 0 and the other determining the actual value if it is non-zero. This idea was first introduced in the econometrics literature to describe health expenditures using a pair of regression equations, one for the probability of expenditure and one for the conditional mean of expenditure  \cite{duan1983, manning1981}. Two-part models have been applied to longitudinal data \cite{olsen2001}, but we do not want to split our response into zeros and non-zeros. Rather, we have a time-ordered outcome response that we believe resulted from two processes, one determining the level and the other determining the pattern of change over time.  In many longitudinal applications, researchers are interested in both shape and level and want to study them simultaneously.


