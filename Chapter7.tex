\chapter{Conclusions}
\label{chap:concl}
Longitudinal data sets include measurements repeated at irregularly intervals over time on a large group of subjects. This highly structure data requires methods that account for the structure and dependence between measurements on the same subject. I focus on cluster analysis methods and study the data generating conditions and for which research questions in which they succeed and fall short. 

\section{Contributions}
This thesis makes several original contributions. The first contribution is my study of covariance misspecification in mixture models in Chapter \ref{chap:misspecify}. When components of a mixture model overlap, it is important to correctly model the within-vector dependence to avoid asymptotic and finite sample bias in the parameter estimates when the number of components are known. Naively assuming conditional independence may change the clustering results leading to incorrect conclusions. This is contrast to most longitudinal data analysis in which the mean can be estimated in an unbiased manner without modeling the inherent dependence. With robust standard errors, one can complete ignore the correlation. Using a working correlation improves the statistical efficient of the estimator. Although not studied in this thesis, this type of misspecification also impacts the choice of the number of optimal components when it is not known a priori. If the dependence structure is more complex than the assumed, more components are needed to model the variability and dependence. This has practical importance when fitting a mixture model to longitudinal data as there is inherent dependence within subjects. This adds to but is consistent with the limited literature on misspecifying mixture in that well separated components are more robust to misspecification. 

When clustering longitudinal data, it is important to consider how two trajectories are deemed similar. Longitudinal studies are typically carried out to directly study the change over time. The second contribution of this thesis is raising awareness that most clustering algorithms do not group individuals based on the shape of the pattern of change of their trajectories.  Rather, most methods including the standard finite mixture model groups individuals on the feature that dominates the variability, which is often the level. The few methods that have been suggested as a way to cluster based on shape are limited in the circumstances in which they flourish.

The third contribution is my adapting and extending standard methods to attempt to cluster based on shape while overcome the shortfalls of the current methodology. The three proposed methods approach the problem from different angles. All of the methods have issues and challenges, but it is clear that vertically shifting mixture models has the best performance when in clustering short noisy trajectories as shown in the simulation study.

Lastly, this thesis attempts to cluster body mass index trajectories by shape and determine the relationship between baseline factors and group membership. The vertically shifting mixture model is fit to the CHAMACOS data and the results are juxtaposed with those from standard clustering methods that are currently used in the literature. In this circumstance where level and shape are moderately related, the group mean curves were similar, but the groups differed in composition enough to impact the inference about baseline factors. PREPREGNANCY IMPACTING LEVEL BUT NOT NECESSARILY THE SHAPE; SOME MODERATE EVIDENCE TO SUGGEST THAT BPA MAY IMPACT SHAPE BUT NOT LEVEL. 

\section{Limitations}
While this thesis attempts to be thorough in its study of these clustering methods, it is not exhaustive. The study about misspecification primarily focuses on wrongly assuming conditional independence when the true data generating dependence for one of the groups is exchangeable with different correlation values and constant variance. The simulation focused on horizontal trends over time assuming that different mean shapes translate into more well separated components in general. To more fully understand the impact of misspecification, one could also use different mean shapes, more components, non-stationary generating covariance structures and varying vector lengths. More suggestions are listed at the end of Chapter \ref{chap:misspecify}. 

The context in which I compare the different methods is a longitudinal data set with only five to ten repeated outcome measurements sparsely observed over time, which is common for the field of public health as it is expensive and time-consuming to perform each annual or biannual interview. Some of the methods may work better on longer trajectories. In that case, if the data is relatively sampled on a dense grid, there may be more options available in the functional data analysis literature. While this may be consider a limitation or shortfall of this thesis, I believe it is worth focusing our attention on this type of data set since it is so common.

In terms of the proposed methods, the vertically shifted mixture models excels in practice even though it has problems theoretically. By subtracting the mean, the transformed vector lies on a subspace of one dimension smaller and thus the covariance matrix of the vertically shifted vector is always singular no matter the original dependence structure. However, we are interested in the covariance of the deviations from the mean for the transformed vectors in the model setting. If the observation times are the same for every individual, the matrix is singular and regularization such as assuming conditional independence or an exponential structure may be necessary for estimation. In this situation, this method may not be able to accurately detect the difference between similar shapes due to the misspecification in the covariance matrix. If the data are observed at random times, the transformation may have a valid covariance matrix. 

It is typical for longitudinal studies to attempt to measure individuals at evenly spaced, regular times. However, many believe there is typically not scientific reasoning for this uniform temporal design \cite{collins2006}. Rather, the temporal design should be based on the the expected trajectory shape; more observations are needed during times of rapid or non-linear change. Although this thesis suggests a random temporal design is preferred when vertically shifting the data, it is important to have enough data points to observe individual trajectories. Therefore, a temporal design should be chosen that is based on the expected shape plus some variability in the actual measurements times between individuals. 

\section{Future work}
The work in this thesis has given rise to many new questions. Mixture models are sensitive to assumptions and incorrectly modeling the covariance structure can lead to bias. For most statistical models, resampling methods can be used to estimate the magnitude of the bias; however, implementation for mixture models is not straightforward. This is an area needing more work as practitioners need some indication that the results are misleading due to bias. 

While this thesis has focused on short time series, it is is not clear which clustering methods in this thesis may perform better when there is more data points per person. Future work should include increasing the number of observations and comparing the three proposed methods. Additionally, as the number of observations increases, functional data analysis methods may be more applicable and should be compared with those in this thesis.

One of the largest contributions of this thesis is separating shape and level into two distinct characteristics of longitudinal trajectories. Research questions usually focus on one of the other so it is important to not muddle the features together especially when doing clustering analysis. It may be worth completing the analysis as a two-part model. The first part would focus on the level and determine which baseline factors impact the level of the trajectory. The second part focuses on the shape and which factors impact the shape group after accounting for level differences. More thought needs to be put into the details of the model and implementation.

I have limited this discussion to only include a basis for time in the explanatory variables but there is nothing preventing us from adding other covariates. However, when clustering based on shape, it does not make sense to include time-fixed covariates that only impacts the vertical level. Time varying covariates could be included; however, more time needs to be spent thinking about the implications on how to interpret the clustering results. 

