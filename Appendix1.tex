\chapter{Standard errors for multivariate Gaussian mixture model}\label{append:1}

Let $\B y_{1},...,\B y_{n}$ be a sample of independent and identically distributed  $m$-variate random vectors drawn from mixture model
$$f(\B y|\B x, \B w, \BS\theta) = \sum^{K}_{k=1}\pi_{k}(\B w, \BS\gamma)f_{k}(\B y|\B x, \BS \theta_{k})$$
where $f_{k}(\B y|\B x, \BS \theta_{k})$ is the density for a multivariate Gaussian with mean $\B x\BS\beta_{k}$ and covariance matrix $\BS\Sigma_{ik}$. We write the log likelihood as
$$L(\BS\theta)=\sum^{n}_{i=1}\log f(\B y_{i}|\B x_{i},\B w_{i},\BS\theta)$$
where the complete parameter vector is $\BS\theta=(\BS\gamma_{1}^{T},...,\BS\gamma_{K-1}^{T},\BS\theta^{T}_{1},...,\BS\theta^{T}_{K})^{T}$ and $\BS\theta_{k}^{T}$ includes mean and covariance parameters. The score vector is defined by $\B q(\BS\theta)=\sum_{i=1}^{n}\B q_{i}(\BS \theta)$, where
$$\B q_{i}(\BS\theta)=\frac{\partial \log f(\B y_{i}|\B x_{i},\B w_{i},\BS\theta)}{\partial \BS\theta}=((\B q^{\gamma_{1}}_{i})^{T},...,(\B q^{\gamma_{K-1}}_{i})^{T},(\B q^{\theta_{1}}_{i})^{T},...,(\B q^{\theta_{K}}_{i})^{T})^{T}$$
and the Hessian matrix defined by $\B Q(\BS\theta)=\sum_{i=1}^{n}\B Q_{i}(\BS\theta)$, where
$$ \B Q_{i}(\BS\theta)=\frac{\partial^{2} \log f(\B y_{i}|\B x_{i},\B w_{i},\BS\theta)}{\partial \BS\theta\partial \BS\theta'}=\left(\begin{array}{cccccc} \B Q^{\gamma_{1}\gamma_{1}}_{i}&\cdots&\B Q^{\gamma_{1}\gamma_{K-1}}_{i}&\B Q^{\gamma_{1}\theta_{1}}_{i}&\cdots&\B Q^{\gamma_{1}\theta_{K}}_{i}\\
\vdots&\ddots&\vdots&\vdots&\ddots&\vdots\\
\B Q^{\gamma_{K-1}\gamma_{1}}_{i}&\cdots&\B Q^{\gamma_{K-1}\gamma_{K-1}}_{i}&\B Q^{\gamma_{K-1}\theta_{1}}_{i}&\cdots&\B Q^{\gamma_{K-1}\theta_{K}}_{i}\\
\B Q^{\theta_{1}\gamma_{1}}_{i}&\cdots&\B Q^{\theta_{1}\gamma_{K-1}}_{i}&\B Q^{\theta_{1}\theta_{1}}_{i}&\cdots&\B Q^{\theta_{1}\theta_{K}}_{i}\\
\vdots&\ddots&\vdots&\vdots&\ddots&\vdots\\
\B Q^{\theta_{K}\gamma_{1}}_{i}&\cdots&\B Q^{\theta_{K}\gamma_{K-1}}_{i}&\B Q^{\theta_{K}\theta_{1}}_{i}&\cdots&\B Q^{\theta_{K}\theta_{K}}_{i}\\
\end{array} \right).$$
Also, we define
\begin{align}
\phi_{ik}=\pi_{k}(\B w_{i},\BS\gamma)f_{k}(\B y_{i}|\B x_{i},\BS\theta_{k}),&\;\; \alpha_{ik}=\frac{\phi_{ik}}{\sum_{k}\phi_{ik}}\label{phi}\\
\B b_{ik}=\BS\Sigma_{ik}^{-1}(\B y_{i}-\B x_{i}\BS\beta_{k}),&\;\; \B B_{ik}=\BS\Sigma_{ik}^{-1}-\B b^{T}_{ik}\B b_{ik},\label{bs2}
\end{align}

\begin{theorem}
For the multivariate Gaussian mixture model, the contribution of the $i$th observation to the score vector with respect to the parameters, $\BS\gamma_{k}$ (k=1,...,K-1) and $\BS\theta_{k}$ $(k=1,..,K)$, is given by
\begin{align*}
\B q_{i}^{\gamma_{k}}=(\alpha_{ik}-\pi_{k}(\B w_{i},\BS\gamma_{k}))\B w_{i},\;\; \B q_{i}^{\theta_{k}}=\alpha_{ik}\B c_{ik}.
\end{align*}
The contribution of the $i$th observation to the Hessian matrix is given by
\begin{align*}
\B Q_{i}^{\gamma_{k}\gamma_{k}}&=[\alpha_{ik}(1-\alpha_{ik})-\pi_{k}(\B w_{i},\BS\gamma_{k})(1-\pi_{k}(\B w_{i},\BS\gamma_{k}))]\B w_{i}\B w_{i}^{T}\\
\B Q_{i}^{\gamma_{k}\gamma_{l}}&=-[\alpha_{ik}\alpha_{il}-\pi_{k}(\B w_{i},\BS\gamma_{k})\pi_{l}(\B w_{i},\BS\gamma_{l})]\B w_{i}\B w_{i}^{T}
\end{align*}
\begin{align*}
\B Q_{i}^{\gamma_{k}\theta_{k}}=\alpha_{ik}(1-\alpha_{ik})\B w_{i}\B c_{ik}^{T},&\quad \B Q_{i}^{\gamma_{k}\theta_{l}}=-\alpha_{ik}\alpha_{il}\B w_{i}\B c_{il}^{T}\\
\B Q_{i}^{\theta_{k}\theta_{k}}=-(\alpha_{ik}\B C_{ik}-\alpha_{ik}(1-\alpha_{ik})\B c_{ik}\B c_{ik}^{T}\B ),&\quad \B Q_{i}^{\theta_{k}\theta_{l}}=-\alpha_{ik}\alpha_{il}\B c_{ik}\B c_{il}^{T}\;\;\;(k\not=l).
\end{align*}
where $\B c_{ik}$ and $\B C_{ik}$ are defined as follows:\\

if $\BS\Sigma_{ik}=\sigma_{k}^{2}I$, then $\BS\theta_{k}=(\BS\beta_{k},\sigma^{2}_{k})$ and
\begin{align*}
\B c_{ik}=\left( \begin{array}{c}\B x_{i}^{T}\B b_{ik}\\ -\frac{1}{2} \B B_{ik}\end{array}\right),&\;\;
\B C_{ik}=\left(\begin{array}{cc} \frac{1}{\sigma^{2}_{k}}\B x_{i}^{T}\B x_{i}&\frac{1}{\sigma^{2}_{k}}\B x_{i}^{T}\B b_{ik}\\ \frac{1}{\sigma^{2}_{k}}\B b^{T}_{ik}\B x_{i}&\frac{1}{2\sigma^{2}_{k}}(\frac{m}{\sigma^{2}_{k}}-2\B B_{ik})\end{array} \right)
\end{align*}

if $\BS\Sigma_{ik}=\sigma_{k}^{2}\B R_{ik}$, then $\BS\theta_{k}=(\BS\beta_{k},\sigma^{2}_{k},\rho_{k})$ and
\begin{align*}
\B c_{ik}&=\left( \begin{array}{c}\B x_{i}^{T}\B b_{ik}\\ -\frac{1}{2}tr( \B B_{ik} \B R_{ik})\\ -\frac{\sigma^{2}_{k}}{2}tr( \B B_{ik} \B R'(\rho_{k}))\end{array}\right),\\
\B C_{ik}&=\left(\begin{array}{ccc} \B x_{i}^{T}\BS\Sigma_{ik}^{-1}\B x_{i}& \B x_{i}^{T}\BS\Sigma_{ik}^{-1}\B R_{ik}\B b_{ik}&\sigma^{2}_{k}\B x_{i}^{T} \BS\Sigma_{ik}^{-1}\B R^{'}_{ik}\B b_{ik}\\  
\B b^{T}_{ik}\B R_{ik}\BS\Sigma_{ik}^{-1}\B x_{i}&\frac{1}{2}tr(\BS \Lambda_{ik}\B R_{ik}\BS\Sigma_{ik}^{-1}\B R_{ik})&\frac{1}{2}tr(\BS \Lambda_{ik}\B R_{ik}\BS\Sigma_{ik}^{-1}\B R^{'}_{ik})\\ 
\sigma^{2}_{k}\B b_{ik}^{T}\B R^{'}_{ik} \BS\Sigma_{ik}^{-1}\B x_{i}&\frac{1}{2}tr(\BS \Lambda_{ik}\B R^{'}_{ik}\BS\Sigma_{ik}^{-1}\B R_{ik})&\frac{1}{2}tr(\BS \Lambda_{ik}\B R^{'}_{ik}\BS\Sigma_{ik}^{-1}\B R^{'}_{ik})
\end{array} \right)
\end{align*}
where $\BS \Lambda_{ik} = \BS\Sigma_{ik}^{-1}-2\B B_{ik}$, $\B R_{ik}$ is a correlation matrix based on the parameter $\rho_{k}$, and $\B R^{'}_{ik}$ is the derivative of $\B R_{ik}$ with respect to $\rho_{k}$. \\

if $\Sigma_{ik}=\nu_{k}^{2}\B J +\sigma_{k}^{2}\B R_{ik}$, then $\BS\theta_{k}=(\BS\beta_{k},\sigma^{2}_{k},\rho_{k},\nu_{k}^{2})$ and
\begin{align*}
\B c_{ik}&=\left( \begin{array}{c}\B x_{i}^{T}\B b_{ik}\\ -\frac{1}{2}tr( \B B_{ik} \B R_{ik})\\ -\frac{\sigma^{2}_{k}}{2}tr( \B B_{ik} \B R'(\rho_{k}))\\
-\frac{1}{2}tr( \B B_{ik} \B J)
\end{array}\right),\\
\B C_{ik}&=\left(\begin{array}{cccc} \B x_{i}^{T}\BS\Sigma_{ik}^{-1}\B x_{i}& \B x_{i}^{T}\BS\Sigma_{ik}^{-1}\B R_{ik}\B b_{ik}&\sigma^{2}_{k}\B x_{i}^{T} \BS\Sigma^{-1}_{ik}\B R^{'}_{ik}\B b_{ik}&\B x_{i}^{T} \BS\Sigma_{ik}^{-1}\B J \B b_{ik}\\  
\B b^{T}_{ik}\B R_{ik}\BS\Sigma_{ik}^{-1}\B x_{i}&\frac{1}{2}tr(\BS \Lambda_{ik}\B R_{ik}\BS\Sigma_{ik}^{-1}\B R_{ik})&\frac{1}{2}tr(\BS \Lambda_{ik}\B R_{ik}\BS\Sigma_{ik}^{-1}\B R^{'}_{ik})&\frac{1}{2}tr(\BS \Lambda_{ik}\B R_{ik}\BS\Sigma_{ik}^{-1}\B J)\\ 
\sigma^{2}_{k}\B b_{ik}^{T}\B R^{'}_{ik} \BS\Sigma_{ik}^{-1}\B x_{i}&\frac{1}{2}tr(\BS \Lambda_{ik}\B R^{'}_{ik}\BS\Sigma_{ik}^{-1}\B R_{ik})&\frac{1}{2}tr(\BS \Lambda_{ik}\B R^{'}_{ik}\BS\Sigma_{ik}^{-1}\B R^{'}_{ik})&\frac{1}{2}tr(\BS \Lambda_{ik}\B R^{'}_{ik}\BS\Sigma_{ik}^{-1}\B J)\\
 \B b_{ik}^{T}\B J \BS\Sigma_{ik}^{-1}\B x_{i}&\frac{1}{2}tr(\BS \Lambda_{ik}\B J\BS\Sigma_{ik}^{-1}\B R_{ik})&\frac{1}{2}tr(\BS \Lambda_{ik}\B J\BS\Sigma_{ik}^{-1}\B R^{'}_{ik})&\frac{1}{2}tr(\BS \Lambda_{ik}\B J\BS\Sigma_{ik}^{-1}\B J)
\end{array} \right)
\end{align*}
where $\B J$ is an $m\times m$ matrix of 1's, $\BS \Lambda_{ik} = \BS\Sigma_{ik}^{-1}-2\B B_{ik}$, $\B R_{ik}$ is a correlation matrix based on the parameter $\rho_{k}$, and $\B R^{'}_{ik}$ is the derivative of $\B R_{ik}$ with respect to $\rho_{k}$. 
\end{theorem}

\begin{proof}
We follow a procedure similar to Boldea and Magnus \cite{boldea2009}. Let $g(\B  z)=\sum_{j=1}^{K}\exp(\B  z^{T}\BS \gamma_{j})$. Then $\pi_{k}(\B w,\BS\gamma)=\frac{\exp( \B  z^{T}\BS\gamma_{k})}{g(\B  z)}$. We will maximize $\pi_{k}(\B w,\BS\gamma)$ with respect to $\BS\gamma_{1},\BS\gamma_{2},...,\BS\gamma_{K-1}$. Taking the logarithm of both sides, we get
\begin{align}
\log \pi_{k}(\B w,\BS\gamma)=\B w^{T}\BS\gamma_{k} - \log g(\B w)
\label{wts}
\end{align}
Note that $\partial\log g(\B w)/\partial \BS \gamma_{j}=\pi_{j}(\B w,\BS\gamma) \B w$. Then,
\begin{align}
\frac{d\log\pi_{k}(\B w,\BS\gamma)}{d\BS\gamma_{j}}&=I(j=k)\B w-\pi_{j}(\B w,\BS\gamma_{j})\B w\quad\text{ for }j=1,...,K-1\nonumber\\
\frac{d^{2}\log\pi_{k}(\B w_{i},\BS\gamma)}{d\BS\gamma_{j}d\BS\gamma_{j}}&=[-I(i=j)\pi_{j}(\B w,\BS\gamma_{j})+\pi_{i}(\B w,\BS\gamma_{i})\pi_{j}(\B w,\BS\gamma_{j})]\B w\B w^{T}\quad\text{ for }i,j=1,...,K-1
\label{logpi}
\end{align}
Let $\phi_{ik}$ and $\alpha_{ik}$ be defined as in \eqref{phi}. Then since $f(\B y_{i}|\B x_{i},\B w_{i},\BS\theta)=\sum^{K}_{k=1}\phi_{ik}$, we obtain
\begin{align}
d\log f(\B y_{i}|\B x_{i},\B w_{i},\BS\theta)=\frac{d f(\B y_{i}|\B x_{i},\B w_{i},\BS\theta)}{f(\B y_{i}|\B x_{i},\B w_{i},\BS\theta)}=\sum^{K}_{k=1}\frac{d\phi_{ik}}{\sum^{K}_{j=1}\phi_{ij}}=\sum^{K}_{k=1}\alpha_{ik}d\log \phi_{ik}\label{dlogf}
\end{align}
and 
\begin{align}\label{d2logf}
d^{2}\log f(\B y_{i}|\B x_{i},\B w_{i},\BS\theta)&=\left(\frac{d^{2} f(\B y_{i}|\B x_{i},\B w_{i},\BS\theta)}{f(\B y_{i}|\B x_{i},\B w_{i},\BS\theta)}-\left(\frac{d f(\B y_{i}|\B x_{i},\B w_{i},\BS\theta)}{f(\B y_{i}|\B x_{i},\B w_{i},\BS\theta)}\right)^{2}\right)\nonumber\\
&=\left(\frac{\sum^{K}_{k=1}d^{2}\phi_{ik}}{\sum^{K}_{k=1}\phi_{ik}}-\left(\frac{\sum^{K}_{k=1}d \phi_{ik}}{\sum^{K}_{k=1}\phi_{ik}}\right)^{2}\right)\nonumber\\
&=\left(\sum^{K}_{k=1}\alpha_{ik}(d^{2}\log \phi_{ik}+(d\log \phi_{ik})^{2})-(\sum^{K}_{k=1}\alpha_{ik}d\log \phi_{ik})^{2}\right)
\end{align}
To evaluate these, we first need the first- and second-order derivatives of $\log \phi_{ik}$.  Since, 
\begin{align*}
\log f_{k}(\B y|\B x,\BS\theta_{k})=-\frac{m}{2}\log(2\pi)-\frac{1}{2}\log |\BS\Sigma_{ik}^{-1}|-(\B y-\B x\BS \beta_{k})^{T}\BS\Sigma_{ik}^{-1}(\B y-\B x\BS\beta_{k})
\end{align*}
we find 
\begin{align*}
d\log f_{k}(\B y|\B x,\BS\theta_{k})&=-\frac{1}{2}d\log|\BS\Sigma_{ik}|+(\B y-\B x\BS \beta_{k})^{T}\BS\Sigma_{ik}^{-1}d(\B x\BS \beta_{k})\\
&\quad-\frac{1}{2}(\B y-\B x\BS \beta_{k})^{T}d\BS\Sigma_{ik}^{-1}(\B y-\B x\BS\beta_{k})\\
&=-\frac{1}{2}tr(\BS\Sigma_{ik}^{-1}d\BS\Sigma_{ik})+(\B y-\B x\BS \beta_{k})^{T}\BS\Sigma_{ik}^{-1}d(\B x\BS \beta_{k})\\
&\quad+\frac{1}{2}(\B y-\B x\BS \beta_{k})^{T}\BS\Sigma_{ik}^{-1}d\BS\Sigma_{ik}\BS\Sigma_{ik}^{-1}(\B y-\B x\BS\beta_{k})\\
\end{align*}
and
\begin{align*}
d^{2}\log f_{k}(\B y|\B x,\BS\theta_{k})&=-\frac{1}{2}tr(d\BS\Sigma_{ik}^{-1}d\BS\Sigma_{ik})-d(\B x\BS \beta_{k})^{T}\BS\Sigma_{ik}^{-1}d(\B x\BS \beta_{k})+(\B y-\B x\BS \beta_{k})^{T}d\BS\Sigma_{ik}^{-1}d(\B x\BS \beta_{k})\\
&\quad-(\B y-\B x\BS \beta_{k})^{T}\BS\Sigma_{ik}^{-1}d\BS\Sigma_{ik}\BS\Sigma_{ik}^{-1}d(\B x\BS\beta_{k})\\
&\quad-(\B y-\B x\BS \beta_{k})^{T}\BS\Sigma_{ik}^{-1}d\BS\Sigma_{ik}\BS\Sigma_{ik}^{-1}d\BS\Sigma_{ik}\BS\Sigma_{ik}^{-1}(\B y-\B x\BS\beta_{k})\\
&=\frac{1}{2}tr(\BS\Sigma_{ik}^{-1}d\BS\Sigma_{ik}\BS\Sigma_{ik}^{-1}d\BS\Sigma_{ik})-d(\B x\BS \beta_{k})^{T}\BS\Sigma_{ik}^{-1}d(\B x\BS \beta_{k})\\
&\quad-2(\B y-\B x\BS \beta_{k})^{T}\BS\Sigma_{ik}^{-1}d\BS\Sigma_{ik}\BS\Sigma_{ik}^{-1}d(\B x\BS\beta_{k})\\
&\quad-(\B y-\B x\BS \beta_{k})^{T}\BS\Sigma_{ik}^{-1}d\BS\Sigma_{ik}\BS\Sigma_{ik}^{-1}d\BS\Sigma_{ik}\BS\Sigma_{ik}^{-1}(\B y-\B x\BS\beta_{k})\\
\end{align*}
and hence, using \eqref{logpi} and the definitions \eqref{phi}-\eqref{bs2},
\begin{align*}
d\log \phi_{ik}&=d\log \pi_{k}(\B w_{i},\BS\gamma_k)+(\B y-\B x\BS \beta_{k})^{T}\BS\Sigma_{ik}^{-1}d(\B x\BS \beta_{k})-\frac{1}{2}tr(\BS\Sigma_{ik}^{-1}d\BS\Sigma_{ik})\\
&\quad+\frac{1}{2}(\B y-\B x\BS \beta_{k})^{T}\BS\Sigma_{ik}^{-1}d\BS\Sigma_{ik}\BS\Sigma_{ik}^{-1}(\B y-\B x\BS\beta_{k})\\
&=\B w_{i} d\BS \gamma_{k} -\pi_{k}(\B w_{i},\BS\gamma_k)\B w_{i}d\BS\gamma_{\cdot}+\B b^{T}_{ik}d(\B x_{i}\BS \beta_{k})-\frac{1}{2}tr(\B B_{ik}d\BS\Sigma_{ik})
\end{align*}
and
\begin{align*}
d^{2}\log \phi_{ik}&=d^{2}\log \pi_{k}(\B w_{i},\BS\gamma_k)-d(\B x\BS \beta_{k})^{T}\BS\Sigma_{ik}^{-1}d(\B x\BS \beta_{k})-2(\B y-\B x\BS \beta_{k})^{T}\BS\Sigma_{ik}^{-1}d\BS\Sigma_{ik}\BS\Sigma_{ik}^{-1}d(\B x\BS\beta_{k})\\
&\quad-(\B y-\B x\BS \beta_{k})^{T}\BS\Sigma_{ik}^{-1}d\BS\Sigma_{ik}\BS\Sigma_{ik}^{-1}d\BS\Sigma_{ik}\BS\Sigma_{ik}^{-1}(\B y-\B x\BS\beta_{k})+\frac{1}{2}tr(\BS\Sigma_{ik}^{-1}d\BS\Sigma_{ik}\BS\Sigma_{ik}^{-1}d\BS\Sigma_{ik})\\
&=-d\BS\gamma_{\cdot}^{T}\pi_{\cdot}(\B w_{i},\BS\gamma_{\cdot})\B w_{i}\B w_{i}^{T}d\BS\gamma_{\cdot}+d\BS\gamma_{\cdot}^{T}\pi_{\cdot}(\B w_{i},\BS\gamma_{\cdot})\pi_{\cdot\cdot}(\B w_{i},\BS\gamma_{\cdot\cdot})\B w_{i}\B w_{i}^{T}d\BS\gamma_{\cdot\cdot}\\
&\quad-d(\B x\BS \beta_{k})^{T}\BS\Sigma_{ik}^{-1}d(\B x\BS \beta_{k})-2\B b_{ik}^{T}(d\BS\Sigma_{ik})\BS\Sigma_{ik}d(\B x\BS\beta_{k})\\
&\quad-\frac{1}{2}tr(\BS\Sigma_{ik}^{-1}-2\B B_{ik})d\BS\Sigma_{ik}\BS\Sigma_{ik}^{-1}d\BS\Sigma_{ik}
\end{align*}
For specific structures of $\BS\Sigma_{ik}$, we can substitute in the derivative:\\

if $\BS\Sigma_{ik}=\sigma_{k}^{2}I$, then $\BS\theta_{k}=(\BS\beta_{k},\sigma^{2}_{k})$ and
$$d\BS\Sigma_{ik}=d\sigma_{k}^{2}I$$

if $\BS\Sigma_{ik}=\sigma_{k}^{2}\B R_{ik}$, then $\BS\theta_{k}=(\BS\beta_{k},\sigma^{2}_{k},\rho_{k})$ and
$$d\BS\Sigma_{ik}=d\sigma_{k}^{2}\B R_{ik} +\sigma_{k}^{2}\B R^{'}_{ik}d\rho_{k}$$

if $\BS\Sigma_{ik}=\nu_{k}^{2}\B J+\sigma_{k}^{2}\B R_{ik}$, then $\BS\theta_{k}=(\BS\beta_{k},\sigma^{2}_{k},\rho_{k})$ and
$$d\BS\Sigma_{ik}=d\nu_{k}^{2}\B J + d\sigma_{k}^{2}\B R_{ik} +\sigma_{k}^{2}\B R^{'}_{ik}d\rho_{k}$$
Using these calculations, we can rewrite the previously equations with the definitions of $\B c_{ik}$ and $\B C_{ik}$ in the the theorem,
\begin{align}\label{dlogphi2}
d\log \phi_{ik}&=\B w_{i} d\BS \gamma_{k} -\pi_{k}(\B w_{i},\BS\gamma_k)\B w_{i}d\BS\gamma_{\cdot}+\B c^{T}_{ik}d\BS\theta_{k} 
\end{align}
and
\begin{align}\label{d2logphi2}
d^{2}\log \phi_{ik}
&=-d\BS\gamma_{\cdot}^{T}\pi_{\cdot}(\B w_{i},\BS\gamma_{\cdot})\B w_{i}\B w_{i}^{T}d\BS\gamma_{\cdot}+d\BS\gamma_{\cdot}^{T}\pi_{\cdot}(\B w_{i},\BS\gamma_{\cdot})\pi_{\cdot\cdot}(\B w_{i},\BS\gamma_{\cdot\cdot})\B w_{i}\B w_{i}^{T}d\BS\gamma_{\cdot\cdot}-d\BS\theta_{k}^{T}\B C_{ik}d\BS\theta_{k}
\end{align}
Inserting \eqref{dlogphi2} in \eqref{dlogf}, and \eqref{d2logphi2} and \eqref{dlogphi2} in \eqref{d2logf} completes the proof.
\end{proof}
By definition, the conventional variance-covariance estimates are $A_{n}(\hat{\BS\theta}_{n})=n^{-1}\sum_{i=1}^{n}\B Q_{i}(\hat{\BS \theta}_{n})$ and $B_{n}(\hat{\BS\theta}_{n})=n^{-1}\sum_{i=1}^{n}\B q_{i}(\hat{\BS\theta}_{n})\B q_{i}^{T}(\hat{\BS\theta}_{n})$. 


